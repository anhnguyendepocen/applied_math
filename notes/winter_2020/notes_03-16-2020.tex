\documentclass{article}
\usepackage[utf8]{inputenc}
\usepackage{amsmath}

\title{Week 8, Lecture 3}
\author{Chandan Tankala}
\date{12 March 2020}

\begin{document}

\maketitle
\section{Diffusions}
Let $X_t\in \mathbb{R}$ be the solution to the SDE 

$$dX_t = a(X_t)dt + b(X_t)dB_t$$

then $X_t$ is called a diffusion process with drift $a(x)$ and diffusion $b(x)$.

\section{Simulation of a diffusion process}

Started from $x$, until $T$:

$$
\begin{align}
    &\text{Let}\;\; X_0 = x \\
    & t= 0 \\
    & \text{while}\;\; t<T: \\
    & \;\;\;\;\;\; W\sim N(0,dt) \\
    & \;\;\;\;\;\; X(t+dt) = X(t) + dta(X_t) + Wb(X_t) \\
\end{align}
$$

Equivalently, if $X_t$ is a time-homogenous Markov process such that:

\begin{align}
    E[X(t+dt)-X(t)] &= a(X_t)dt+\Theta(dt^2) \\
    Var[X(t+dt)-X(t)] &= b(X(t))^2dt+\Theta(dt^2)
\end{align}

Then:

$$X_t = X_0 + \int_0^ta(X_s)ds+\int_0^tb(X_s)dB_s$$

Recall the example: \\

\begin{align}
    X_t &= B_t^2 \\
    dX_t &= 2B_tdB_t + dt \\
    B_t^2 &= B_0^2 + \int_0^t 2B_sdB_s + t
\end{align}

\section{Example}

ODE: Let us say we solve $f(t) = 2\sqrt{f(t)}$, then $f(t) = f(0)+t^2$ \\
SDE: Let us say we solve $dX_t = 2\sqrt{X_t}dB_t+dt$, then $X_t = B_t^2+X_0$ is the unique solution.

\section{Theorems}
Let $p(t,x,y)= P_x\{X_t=y\}$. Then this is the unique solution to $\frac{d}{dt}p(t,x,y) = a(x)\frac{d}{dx}+\frac{1}{2}b(x)^2\frac{d^2}{dx^2}p(t,x,y)$ with the boundary condition $p(t,x,y)\xrightarrow{t\to 0}\delta_x(y)$. \\

Recall: For $B_t$, \\
$$p(t,x,y) = \frac{1}{\sqrt{2\pi t}}e^{-\frac{(x-y)^2}{2t}}$$ solves the equation $$\frac{d}{dt} p(t,x,y) = \frac{1}{2}\frac{d^2}{dx^2}p(t,x,y)$$

Definition: The generator a CTMP $X_t\in S$ is $G:f\to Gf$ with $f: S\to \mathbb{R}, GF:S\to \mathbb{R}$ is defined as follows:

$$Gf(x) = \lim_{\epsilon \to 0}\frac{E_x[F(X_\epsilon)-f(x)]}{\epsilon}$$

Example: If $X_t$ is a CTMC on $\{1,2\cdots n\} = S$. So if $f: S\to \mathbb{R}$, then $f\in \mathbb{R}^n$ and $G\in \mathbb{R}^{n\times n}$ with 

$$
G_{ij}=
\begin{cases}
\textbf{jump rates if $i\neq j$}\\
\textbf{- Sum of jump rates if $i=j$}\\
\end{cases}
$$

Theorem: $\frac{d}{dt}p(t,x,y) = G_x p(t,x,y)$ \\
Proof: For diffusions, let $dX = a(X)dt+b(X)dB$, $G = a(x)\frac{d}{dx}+\frac{1}{2}b(x)^2\frac{d^2}{dx^2}$. Let $F_t(x) = E_x[f(X_t)]$. Then:

\begin{align}
    F_{t+\epsilon}(x) &= E_x[f(X_{t+\epsilon})] \\
    &= E_x[E[f(X_{t+\epsilon})|X_{\epsilon}]] \\
    &= E_x[F_t(X_\epsilon)] \textbf{By time homogeneity and Markov property} \\
\end{align} 

So
\begin{align}
    \frac{d}{dt}F_t(x) &= \lim_{\epsilon\to 0} \frac{F_{t+\epsilon}(x)-F(x)}{\epsilon} \\
    &=  \lim_{\epsilon\to 0} \frac{E_x[F_t(X_\epsilon)]-F(x)}{\epsilon} \\
    &= GF(x) \textbf{ By definition of $G$} 
\end{align}

So

\begin{align}
    F_t(x) &= E_x[f(X_y)] \\
    &= \int_{\mathbb{R}} p(t,x,y)f(y)dy \\
    \implies \frac{d}{dt}F_t(x) &= \int_{\mathbb{R}}\frac{d}{dt}p(t,x,y)f(y)dy \\
    \implies Gf_t(x) &= \int_{\mathbb{R}} \left( a(x)\frac{d}{dx} + \frac{1}{2}b(x)^2\frac{d^2}{dx^2}\right) p(t,x,y)f(y)dy
\end{align}

This is true for all $f$, so $\frac{d}{dt}p(t,x,y) = G_x p(t,x,y)$


\end{document}
