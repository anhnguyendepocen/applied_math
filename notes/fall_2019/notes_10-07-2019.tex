\documentclass[12pt,letterpaper]{article}
\usepackage[latin1]{inputenc}
\usepackage[T1]{fontenc}
\usepackage{amsmath}
\usepackage{amsfonts}
\usepackage{amssymb}
\usepackage{graphicx}
\usepackage{amsthm}
\usepackage{tikz}
\theoremstyle{definition}
\newtheorem{example}{Example}[section]
\begin{document}
	\begin{example}[Enumerative Counting Problems]
		Let $f(n)$ be the number of $n$-step paths starting at the origin and taking non-intersecting steps in the directions $(0,1)$, $(-1,0)$, or $(1,0)$, labeled $N$, $W$, and $E$ respectively.
		
		\begin{tikzpicture}
			\draw (0,5)--(0,-1);
			\draw (-3,0)--(5,0);
			\draw[thick, color = red] (0,0)--(0,1)--(-1,1)--(-1,2)--(0,2)--(1,2)--(2,2)--(2,3)--(3,3)--(3,4)--(2,4)--(1,4);
		\end{tikzpicture}  in 11 steps
		
		Then we can ditch the picture and instead count strings of $\{N,W,E\}$-letters where $EW$ or $WE$ does not appear as a factor. 
		
		Let's try to express $f(n)$ in terms of $f(k)$ for $k < n$, trying for polynomial coefficients. We can start from the end. A word of length $n$ ends in: $$\begin{cases}
		N & f(n-1) \text{ of these}\\
		NW & f(n-2) \text{ of these}\\
		NE\\
		WW & f(n-1) \text{ of these } NE, WW, \text{ and } EE \text{ combined}\\
		EE
		\end{cases}$$
		Then $\star N$ becomes $\star NE$, $\star W$ becomes $\star WW$, and $\star E$ becomes $EE$ when at the end of a word to give the three $NE, WW, EE$ cases becoming $f(n-1)$. 
		
		Alternatively, we can brute force it to get the first fer tewms to find a recurrence with linear algebra. Then $f(n) = 2f(n-1) + f(n-2)$ where $f(0) = 1$ and $f(1) = 3$. Then we have \[\sum_{n\geq 2}f(n)x^n = 2\sum_{n\geq2} f(n-1)x^n + \sum_{n\geq 2} f(n-2)x^n.\] Note that we can put $G(x) = \sum_{n\geq 0} f(n) x^n = 1 + 3x + \ldots$, so \[G(x) - 1 - 3x = \sum_{n\geq 2}f(n)x^n, 2x[G(x) - 1] = 2\sum_{n\geq2} f(n-1)x^n, \text{ and } x^2G(x) = \sum_{n\geq 2} f(n-2)x^n\] so we can rewrite our equation as \[G(x) - 1 - 3x = 2x[G(x) - 1] + x^2 G(x).\]
		
		Then we can do some algebra to get \[G(x) = \frac{1+x}{1-2x-x^2} = \frac{1+x}{(1-(1+\sqrt{2})x)(1-(1-\sqrt{2})x)} = \frac{A}{1-(1+\sqrt{2})x} + \frac{B}{1-(1-\sqrt{2})x}\]
		
		Next clear the denominators and compare coefficients and expand these geometric series to get \[G(x) = \sum_{n\geq0}\left[\frac{1}{2}(1+\sqrt{2})^{n+1} + \frac{1}{2}(1-\sqrt{2})^{n+1}\right] x^2.\]
		Note that $1+\sqrt{2} \approx 2.414$ and $1- \sqrt{2} \approx -0.414 < 1$, so $f(n) = \frac{1}{2}(1+\sqrt{2})^{n+1} + o(1)$. 

	\end{example}		
In general, growth of the terms $\{a_n\}$ depends on the closest pole of $G(x) = \sum a_nx^n$ to the origin, if $G(x)$ has isolated singularities. 

Remark: $G(x) = \frac{p(x)}{q(x)} = \sum f(n)x^n$. Then there should be a linear recurrence for $f(n)$. If $q$ has distinct roots, then recurrence has constant coefficients. There should be an explicit formula for $f(n)$ in terms of the roots of $q(x)$. 
\end{document}